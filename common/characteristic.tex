{\actuality}
Существуют области применения где шагающие роботы обладают преимуществами перед колесными роботами, такие как перемещение в условиях техногенных завалов, движение по металлоконструкциям и трубам, преодоление препятствий характерный размер которых равен или больше размеров робота, движение по слабым опорным поверхностям - лед и снег, движение по кочкам болота, залезание на деревья и столбы. Многоногие машины обладают запасом устойчивости в отличие от двуногих роботов.

В работах \todo{[]} исследуется построение управления при помощи заранее заданных шаговых циклов и набору следовых точек. Выполняется решение задачи преодолевать фиксированный тип препятствий или их комбинацию \todo{[корянов/голубев]}. 

Количество работ, использующих методы из области машинного обучения растет. Появление специализированного программного обеспечения и аппаратных решений упрощает и ускоряет процесс обучения нейронных сетей. В работах \todo{[]} исследуется построение системы управления на основе нейронных сетей при помощи алгоритмов обратного распространения при известном обучающем множестве. В работах \todo{[]} описывается обучение нейронной сети при помощи обучения с подкреплением.

В большей части работ \todo{[]} рассматривается случай, когда робот представлен в виде цельного корпуса и шести ног, расположенных симметрично от центра корпуса или его продольной оси.



%Обзор, введение в тему, обозначение места данной работы в
%мировых исследованиях и~т.\:п., можно использовать ссылки на~другие
%работы\ifnumequal{\value{bibliosel}}{1}{~\autocite{Gosele1999161}}{}
%(если их~нет, то~в~автореферате
%автоматически пропадёт раздел <<Список литературы>>). Внимание! Ссылки
%на~другие работы в разделе общей характеристики работы можно
%использовать только при использовании \verb!biblatex! (из-за технических
%ограничений \verb!bibtex8!. Это связано с тем, что одна
%и~та~же~характеристика используются и~в~тексте диссертации, и в
%автореферате. В~последнем, согласно ГОСТ, должен присутствовать список
%работ автора по~теме диссертации, а~\verb!bibtex8! не~умеет выводить в одном
%файле два списка литературы).
%При использовании \verb!biblatex! возможно использование исключительно
%в~автореферате подстрочных ссылок
%для других работ командой \verb!\autocite!, а~также цитирование
%собственных работ командой \verb!\cite!. Для этого в~файле
%\verb!Synopsis/setup.tex! необходимо присвоить положительное значение
%счётчику \verb!\setcounter{usefootcite}{1}!.
%
%Для генерации содержимого титульного листа автореферата, диссертации
%и~презентации используются данные из файла \verb!common/data.tex!. Если,
%например, вы меняете название диссертации, то оно автоматически
%появится в~итоговых файлах после очередного запуска \LaTeX. Согласно
%ГОСТ 7.0.11-2011 <<5.1.1 Титульный лист является первой страницей
%диссертации, служит источником информации, необходимой для обработки и
%поиска документа>>. Наличие логотипа организации на титульном листе
%упрощает обработку и поиск, для этого разметите логотип вашей
%организации в папке images в формате PDF (лучше найти его в векторном
%варианте, чтобы он хорошо смотрелся при печати) под именем
%\verb!logo.pdf!. Настроить размер изображения с логотипом можно
%в~соответствующих местах файлов \verb!title.tex!  отдельно для
%диссертации и автореферата. Если вам логотип не~нужен, то просто
%удалите файл с логотипом.
%
%\ifsynopsis
%Этот абзац появляется только в~автореферате.
%Для формирования блоков, которые будут обрабатываться только в~автореферате,
%заведена проверка условия \verb!\!\verb!ifsynopsis!.
%Значение условия задаётся в~основном файле документа (\verb!synopsis.tex! для
%автореферата).
%\else
%Этот абзац появляется только в~диссертации.
%Через проверку условия \verb!\!\verb!ifsynopsis!, задаваемого в~основном файле
%документа (\verb!dissertation.tex! для диссертации), можно сделать новую
%команду, обеспечивающую появление цитаты в~диссертации, но~не~в~автореферате.
%\fi
%
%% {\progress} 
%% Этот раздел должен быть отдельным структурным элементом по
%% ГОСТ, но он, как правило, включается в описание актуальности
%% темы. Нужен он отдельным структурынм элемементом или нет ---
%% смотрите другие диссертации вашего совета, скорее всего не нужен.

{\aim} данной работы является построение алгоритма управления движением робота с артикулированным корпусом при прохождении полосы препятствий. Корпус робота состоит из трех и более звеньев, соединенных вращательными шарнирами. Полоса препятствий состоит из рва, вала, узкого прохода в стене, участка с поврежденным рельефом. Определении технических характеристик робота, необходимых для выполнения построенных движений.

Для~достижения поставленной цели необходимо было решить следующие {\tasks}:
\begin{enumerate}
  \item Исследовать обратную задачу кинематики для многозвенных корпусов
  \item Разработать математическую модель полосы препятствий и робота
  \item Разработать архитектуру системы управления роботом в том числе его подсистем
  \item Разработать программные компоненты управления корпусом, ногами, движением центра масс робота
  \item Выполнить верификацию разработанного алгоритма на виртуальной модели полигона
\end{enumerate}


{\novelty}
\begin{enumerate}
  \item Заключается в предложенных и отработанных в виртуальной среде алгоритмах и библиотеках построения системы управления многоногого шагающего робота с артикулированным корпусом.
  \item Оригинальное исследование кинематики так называемого мозаичного корпуса робота.
\end{enumerate}

{\influence} Полученные результаты могут быть использованы при проектировании и изготовлении системы управления многоногих роботов с артикулированным корпусом. Разработанная библиотека представляют самостоятельную ценность и может быть использованы при изучении более сложных конструкций корпуса многоногих роботов.

{\methods} 
В программном комплексе Универсальный Механизм разработана математическая модель динамики робота, полосы препятствий, взаимодействия робота с препятствиями. Вне программного комплекса Универсальный Механизм разработаны программные компоненты системы управлениям движением.


%{\defpositions}
%\begin{enumerate}
%  \item \todo{Первое положение}
%  \item \todo{Второе положение}
%  \item Третье положение
%  \item Четвертое положение
%\end{enumerate}

%В папке Documents можно ознакомиться в решением совета из Томского ГУ
%в~файле \verb+Def_positions.pdf+, где обоснованно даются рекомендации
%по~формулировкам защищаемых положений. 

{\reliability} полученных результатов обеспечивается корректностью построенной математической моделью робота и результатами численного моделирования, независимого тестирования среды Универсальный Механизм. Результаты находятся в соответствии с результатами, полученными другими авторами.

{\probation}
Основные результаты работы докладывались~на:
перечисление основных конференций, симпозиумов и~т.\:п.

\begin{enumerate}
  \item \todo{добавить}
  \item CLAWAR 2015 \todo{уточнить}
  \item МИКМУС 2016 \todo{уточнить}
  \item МИКМУС 2017 \todo{уточнить}
  \item CLAWAR 2018 \todo{уточнить}
\end{enumerate}

{\contribution} Автор принимал активное участие \ldots

%\publications\ Основные результаты по теме диссертации изложены в ХХ печатных изданиях~\cite{Sokolov,Gaidaenko,Lermontov,Management},
%Х из которых изданы в журналах, рекомендованных ВАК~\cite{Sokolov,Gaidaenko}, 
%ХХ --- в тезисах докладов~\cite{Lermontov,Management}.

\ifnumequal{\value{bibliosel}}{0}{% Встроенная реализация с загрузкой файла через движок bibtex8
    \publications\ Основные результаты по теме диссертации изложены в XX печатных изданиях, 
    X из которых изданы в журналах, рекомендованных ВАК, 
    X "--- в тезисах докладов.%
}{% Реализация пакетом biblatex через движок biber
%Сделана отдельная секция, чтобы не отображались в списке цитированных материалов
    \begin{refsection}[vak,papers,conf]% Подсчет и нумерация авторских работ. Засчитываются только те, которые были прописаны внутри \nocite{}.
        %Чтобы сменить порядок разделов в сгрупированном списке литературы необходимо перетасовать следующие три строчки, а также команды в разделе \newcommand*{\insertbiblioauthorgrouped} в файле biblio/biblatex.tex
        \printbibliography[heading=countauthorvak, env=countauthorvak, keyword=biblioauthorvak, section=1]%
        \printbibliography[heading=countauthorconf, env=countauthorconf, keyword=biblioauthorconf, section=1]%
        \printbibliography[heading=countauthornotvak, env=countauthornotvak, keyword=biblioauthornotvak, section=1]%
        \printbibliography[heading=countauthor, env=countauthor, keyword=biblioauthor, section=1]%
        \nocite{%Порядок перечисления в этом блоке определяет порядок вывода в списке публикаций автора
                vakbib1,vakbib2,%
                confbib1,confbib2,%
                bib1,bib2,%
        }%
        \publications\ Основные результаты по теме диссертации изложены в~\arabic{citeauthor}~печатных изданиях, 
        \arabic{citeauthorvak} из которых изданы в журналах, рекомендованных ВАК, 
        \arabic{citeauthorconf} "--- в~тезисах докладов.
    \end{refsection}
    \begin{refsection}[vak,papers,conf]%Блок, позволяющий отобрать из всех работ автора наиболее значимые, и только их вывести в автореферате, но считать в блоке выше общее число работ
        \printbibliography[heading=countauthorvak, env=countauthorvak, keyword=biblioauthorvak, section=2]%
        \printbibliography[heading=countauthornotvak, env=countauthornotvak, keyword=biblioauthornotvak, section=2]%
        \printbibliography[heading=countauthorconf, env=countauthorconf, keyword=biblioauthorconf, section=2]%
        \printbibliography[heading=countauthor, env=countauthor, keyword=biblioauthor, section=2]%
        \nocite{vakbib2}%vak
        \nocite{bib1}%notvak
        \nocite{confbib1}%conf
    \end{refsection}
}
При использовании пакета \verb!biblatex! для автоматического подсчёта
количества публикаций автора по теме диссертации, необходимо
их~здесь перечислить с использованием команды \verb!\nocite!.
