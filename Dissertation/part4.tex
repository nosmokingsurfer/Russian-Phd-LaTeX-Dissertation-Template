
\chapter{Многоногий шагающий аппарат с трёхзвенный корпусом}
\section{Описание трехзвенного робота}

Рассмотрим робот с трехзвенным корпусом. Звенья корпуса соединены между собой при помощи одностепенных вращательных шарниров. 

\todo{картинка кинематической схемы робота}


\subsection{Преодоление препятствия типа уступ. Постановка задачи}

\todo{Вставить картинку}

Препятствие типа "Прямой уступ" состоит из трёх контактных площадок - две горизонтальные площадки и одна вертикальная. Расстояние по вертикали между двумя горизонтальными площадками $h$. В начальный момент времени робот стоит на нижней площадке, на некотором расстоянии от вертикальной стенки. Вертикальная стенка образует прямой угол с горизонтальными площадками.

Считаем, что робот двигается в режиме статической устойчивости и не менее трех ног находятся в контакте с опорной поверхностью. Работ двигается, так называемой, походкой галоп - поочередно переставляя пару симметричных ноги по левому и правому борту. Робот двигается достаточно медленно.

При залезании на верхнюю горизонтальную площадку прямого уступам можно выделить следующие конфигурации расположения ног робота:

\begin{enumerate}
  \item все ноги робота опираются на нижнюю горизонтальную площадку
  \item одна пара ног опирается на вертикальную площадку, а другая пара ног опирается на нижнюю горизонтальную площадку
  \item одна пара ног опирается на верхнюю контактную площадку, а другая пара ног опирается на вертикальную площадку
  \item все ноги робота опираются на верхнюю горизонтальную площадку уступа
\end{enumerate}


%Первый этап --- подход аппарата к вертикальной площадке и зацепление передними ногами за верхнюю кромку препятствия. В конце первого этапа корпус аппарата принимает вертикальное положение, передние ноги находятся в контакте с верхней площадкой, остальные ноги опираются на вертикальную площадку. Второй этап --- последовательное подтягивание корпуса и перемещение сегментов через верхнюю кромку препятствия. В конце второго этапа корпус робота принимает горизонтальное положение, все ноги аппарата находятся в контакте с верхней горизонтальной площадкой.
%
%Аппарат перемещается при помощи походки "галоп", т.е. переносятся симметричные пары ног --- две передние ноги, две средние, две задние. Будем считать, что аппарат двигается достаточно медленно. 

Исследуем статическую устойчивость робота для каждой конфигурации расположения ног на опорных площадках препятствия. 


\subsection{Первая конфигурация}

%В первом этапе возможны четыре ситуации расположения ног по контактным площадкам. (\textbf{Список ниже относится к этому предложению? - Там пунктов больше чем 4. Если не к нему, тогда это описание первого этапа (ведь так?) и это надо обозначить. А также выделить отдельно принципиальное различие между 4 ситуациями, ведь дальше ты начинаешь разбирать их по порядку, а не понятно, что есть первый случай})

%\begin{itemize}
%\item Передние ноги переносятся на вертикальную стенку. В фазе переноса робот опирается на средние и задние ноги.
%\item Аппарат смещает корпус ближе к стенке. Средние и задние ноги переносятся ближе к стенке, но остаются на нижней площадке.
%\item Средние ноги переносятся на вертикальную стенку. В фазе переноса аппарат опирается на передние и задние ноги.
%\item Аппарат поочередно переставляет задние ноги ближе к вертикальной площадке, чтобы можно было дотянутся передними ногами до верхней кромки препятствия. В фазе переноса робот аппарат опирается на передние и средние ноги, а так же на одну из задних ног. Чтобы избежать опрокидывания вправо или влево требуется дополнительно смещать корпус в поперечном направлении.
%\item Передние ноги переносятся на верхнюю площадку. Опора происходит на средние и задние ноги.
%\item Задние ноги переносятся на вертикальную площадку. Аппарат висит на передних ногах и упирается средними ногами в вертикальную площадку.
%\end{itemize}


Исследуем на устойчивость конфигурацию, когда одна пара ног робота опирается на вертикальную стенку, а другая пара ног опирается на нижнюю горизонтальную площадку. Робот будет находится в статическом равновесии, когда сумма всех внешних сил и сумма всех внешних моментов будут равны нулю.

Введём неподвижную систему координат $Oxyz$. Оси системы направлены как показано на рис.(\todo{вставить ссылку}).

Пусть точки опоры ног имеют координаты:

\begin{equation}
  \label{eq:step_points}
  \begin{alignedat}{3}
    &\overline{r}_1 &= (d, 0, h) , &\overline{r}_2 &= (-d, 0, h) \\
    &\overline{r}_3 &= (d, l, 0) , &\overline{r}_4 &= (-d, l, 0) \\
  \end{alignedat}
\end{equation}


Положение центра масс аппарата задано вектором:

\[\overline{r}_c = (0, y_c, z_c)\]

На центр масс робота действует сила тяжести:

\begin{equation}
\label{eq:F_c}
  \overline{P} = (0,0,-P)
\end{equation}

В точках контакта ног с опорной поверхностью, на ноги действуют реакции опоры:

\begin{equation}
\label{eq:reactions}
  \overline{R}_i := N^i\cdot\overline{n}_i + F_\tau^i\cdot\overline{\tau}_i + F_\nu^i\cdot\overline{\nu}_i = \overline{N}^i + \overline{F}_\tau^i + \overline{F}_\nu^i,
\end{equation}

где $N_i$ - модуль нормальной составляющей $i$-й реакции, направленный по нормали $\overline{n}_i$, $F_\tau^i$ и $F_\nu^i$ - проекции касательной составляющей реакции, в проекции на локальные оси $\overline{\tau}_i$ и $\overline{\nu}_i$.

Чтобы система находилась в положении равновесия сумма всех внешних сил должна быть равна нулю, а также сумма всех моментов внешних сил относительно любой точки должна быть равна нулю:

\begin{equation}
\label{eq:general_static_equations}
  \left\{
    \begin{alignedat}{3}
      &\sum_{i=1}^4\overline{R}_i+\overline{P} = 0, \\
      &\sum_{i=1}^4[\overline{r}_i\times\overline{R}_i] + [\overline{r}_c\times\overline{P}] = 0\\
    \end{alignedat}
  \right.
\end{equation}



Распишем все слагаемые из уравнения \ref{eq:reactions}.

Нормальные составляющие $\overline{N}^i$ реакций опоры:

\begin{equation}
\label{eq:N_i}
  \begin{alignedat}{3}
    &\overline{N}^1 &= &N_1\cdot(0,1,0),  \\
    &\overline{N}^2 &= &N_2\cdot(0,1,0),  \\
    &\overline{N}^3 &= &N_3\cdot(0,0,1),  \\
    &\overline{N}^4 &= &N_4\cdot(0,0,1).  \\
  \end{alignedat}
\end{equation}

Соответствующие силы трения в точках контакта:
\begin{equation}
\label{eq:F_tau_i}
  \begin{alignedat}{3}
    &\overline{F}_\tau^1 &= &F_\tau^1\cdot(0,0,1),  \\
    &\overline{F}_\tau^2 &= &F_\tau^2\cdot(0,0,1),  \\
    &\overline{F}_\tau^3 &= &F_\tau^3\cdot(0,1,0),  \\
    &\overline{F}_\tau^4 &= &F_\tau^4\cdot(0,1,0).  \\
  \end{alignedat}
\end{equation}

\begin{equation}
\label{eq:F_nu_i}
  \begin{alignedat}{1}
  \overline{F}_\nu^1 = F_\nu^1\cdot(1,0,0), \\
  \overline{F}_\nu^2 = F_\nu^2\cdot(1,0,0), \\
  \overline{F}_\nu^3 = F_\nu^3\cdot(1,0,0), \\
  \overline{F}_\nu^4 = F_\nu^4\cdot(1,0,0). \\
  \end{alignedat}
\end{equation}

Рассчитаем моменты сил для реакций опоры и силы тяжести относительно начала системы координат $Oxyz$.

\begin{equation}
\label{eq:momentums}
  \begin{alignedat}{3}
    &[\overline{r}_1\times\overline{R}_1] &=&  (d,0,h)\times(F_\nu^1, N^1, F_\tau^1) &=& (-N^1h,F_\nu^1h - F_\tau^1d,N^1d),  \\
    &[\overline{r}_2\times\overline{R}_2] &=& (-d,0,h)\times(F_\nu^2, N^2, F_\tau^2) &=& (-N^2h,F_\nu^2h + F_\tau^2d,-N^2d), \\
    &[\overline{r}_3\times\overline{R}_3] &=&  (d,l,0)\times(F_\nu^3, F_\tau^3, N^3) &=& (N^3l,-N^3d, F_\tau^3d-F_\nu^3l), \\
    &[\overline{r}_4\times\overline{R}_4] &=& (-d,l,0)\times(F_\nu^4, F_\tau^4, N^4) &=& (N^4l, N^4d,-F_\tau^4d-F_\nu^4l),\\
    &[\overline{r}_c\times\overline{P}]   &=&  (0, y_c, z_c)\times(0,0,-P) &=& (-Py_c,0,0)\\
  \end{alignedat}
\end{equation}

Подставим полученные выражения для сил \ref{eq:N_i}, \ref{eq:F_nu_i}, \ref{eq:F_tau_i}, \ref{eq:F_c} и моментов \ref{eq:momentums} в уравнения статического равновесия (\ref{eq:general_static_equations}) и получим следующую систему из шести уравнений:

\begin{equation}
  \label{eq:case1_initial}
  \left\{
    \begin{alignedat}{3}  
      %Fnu_1 + Fnu_2 + Fnu_3 + Fnu_4 == 0
      %Ft_3 + Ft_4 + N_1 + N_2 == 0
      %Ft_1 + Ft_2 + N_3 + N_4 == P  
      F_\nu^1 + F_\nu^2 + F_\nu^3 + F_\nu^4 = 0, \\
      N^1 + N^2 + F_\tau^3 + F_\tau^4 = 0, \\
      F_\tau^1 + F_\tau^2 + N^3 + N^4 = P, \\
      %  N_1 h + N_2 h + P y_c == l (N_3 + N_4)
      %  d (Ft_1 + N_3) == Ft_2 d + Fnu_1 h + Fnu_2 h + N_4 d  
      %  d (Ft_3 + N_1) == Ft_4 d + N_2 d + Fnu_3 l + Fnu_4 l  
      N^1h + N^2h + Py_c = l(N^3 + N^4), \\
      d(F_\tau^1 + N^3) = F_\tau^2d + F_\nu^1h + F_\nu^2h + N^4d, \\
      d(F_\tau^3 + N^1) = F_\tau^4d + N^2d +F_\nu^3l + F_\nu^4l.\\
    \end{alignedat}
  \right.
\end{equation}



Используем модель кулоновского трения и выполним замены $F_\tau^i = k_\tau^i\cdot N^i$ и $F_\nu^i = k_\nu^i\cdot N^i$ в уравнении \ref{eq:case1_initial}. Получим следующий вид уравнений:

\begin{equation}
\label{eq:case1_subs_kulon}
\left\{
  \begin{alignedat}{3}  
    %N_1 knu_1 + N_2 knu_2 + N_3 knu_3 + N_4 knu_4 == 0
    N^1k_\nu^1 + N^2k_\nu^2 + N^3k_\nu^3 + N^4k_\nu^4 = 0,\\
    %N_1 + N_2 + N_3 kt_3 + N_4 kt_4 == 0
    N^1 + N^2 + N^3k_\tau^3 + N^4k_\tau^4 = 0, \\
    %N_3 + N_4 + N_1 kt_1 + N_2 kt_2 == P
    N^3 + N^4 + N^1k_\tau^1 + N^2k_\tau^2 = P, \\
    %N_1 h + N_2 h + P y_c == l (N_3 + N_4)
    N^1h + N^2h + Py_c = l(N^3 + N^4), \\
    %d (N_3 + N_1 kt_1) == N_4 d + N_2 d kt_2 + N_1 h knu_1 + N_2 h knu_2
    d(N^3 + N^1k_\tau^1) = dN^4 + dN^2k_\tau^2 + hN^1k_\nu^1 + hN^2k_\nu^2,\\
    %d (N_1 + N_3 kt_3) == N_2 d + N_4 d kt_4 + N_3 knu_3 l + N_4 knu_4 l
    d(N^1 + N^3k_\tau^3) = dN^2 + dN^4k_\tau^4 + lN^3k_\nu^3 + lN^4k_\nu^4\\
  \end{alignedat}
\right.
\end{equation}


В полученной системе \ref{eq:case1_subs_kulon} из шести уравнений содержится двенадцать неизвестных:
$k_\nu^1, k_\nu^2, k_\nu^3, k_\nu^4, k_\tau^1, k_\tau^2, k_\tau^3, k_\tau^4, N_1, N_2, N_3, N_4$ --- неизвестные, причём $N_i>0$. Введем дополнительное соотношение на коэффициенты трения:

\begin{equation}
  \label{eq:case1_assumption_1}
  \left\{
    \begin{alignedat}{3}
      k_\nu^1  &= -&k_\nu^2& &= k_\nu,\\
      k_\nu^3  &= -&k_\nu^4& &= k_\nu,\\
      k_\tau^1 &= &k_\tau^2& &= k_\tau^u,\\
      k_\tau^3 &= &k_\tau^4& &= k_\tau^d,\\
    \end{alignedat}
  \right.
\end{equation}

После подстановки дополнительных соотношений \ref{eq:case1_assumption_1} в \ref{eq:case1_subs_kulon} получим следующую систему уравнений:

\begin{equation}
\label{eq:case1_subs_assumption_1}
  \left\{
    \begin{alignedat}{3}
      k_\nu(N^1 + N^3 - N^2 - N^4) = 0,\\
      N^1 + N^2 + k_\tau^d(N^3 + N^4) = 0,\\
      k_\tau^u(N^1 + N^2) + N^3 + N^4 = P,\\
      N^1h + N^2h + Py_c = l(N^3 + N^4), \\      
      %N_3 d + N_1 d kt_u + N_2 h knu == N_4 d + N_2 d kt_u + N_1 h knu  
      dN^3 + dN^1k_\tau^u + hN^2k_\nu = dN^4 + dN^2k_\tau^u + hN^1k_\nu,\\
      %N_1 d + N_3 d kt_d + N_4 knu l == N_2 d + N_4 d kt_d + N_3 knu l
      dN^1 + dN^3k_\tau^d + lN^4k_\nu = dN^2 + dN^4k_\tau^d + lN^3k_\nu.\\
    \end{alignedat}
  \right.
\end{equation}

В системе \ref{eq:case1_subs_assumption_1} из шести уравнений уже семь неизвестных. Дополнительно введем предположение что правые и левые ноги одинаково нагружены, т.е.:

\begin{equation}
\label{eq:case1_assumption_2}
  \left\{
    \begin{alignedat}{3}
    N^1 = N^3 = N^u, \\
    N^2 = N^4 = N^d. \\
    \end{alignedat}
  \right.
\end{equation}


Подставим соотношение \ref{eq:case1_assumption_2} в систему \ref{eq:case1_subs_assumption_1}. В результате первое, пятое и шестое уравнение \ref{eq:case1_subs_assumption_1} превратятся в тождества:

\begin{equation}
\label{eq:case1_subs_assumption_2}
  \left\{
  \begin{alignedat}{3}
    % N_u + N_d kt_d == 0
    N^u + N^dk_\tau^d = 0, \\
    %2 N_d + 2 N_u kt_u == P
    2N^d + 2N^uk_\tau^u = P, \\
    %2 N_u h + P y_c == 2 N_d l
    2hN^u + Py_c = 2lN^d.\\
  \end{alignedat}
  \right.
\end{equation}

В системе \ref{eq:case1_subs_assumption_2} уже три уравнения и четыре неизвестных. Введем допущение, что коэффициенты трения для передних и задних ног совпадают: 

\begin{equation}
  \label{eq:case1_assumption_3}
  k_\tau^u = k_\tau^d = k_\tau > 0
\end{equation}


Подставив \ref{eq:case1_assumption_3} в \ref{eq:case1_subs_assumption_2} в итоге получим:

\begin{equation}
  \label{eq:case1_final_system}
  \left\{
  \begin{alignedat}{2}
%    N_u == N_d kt
    N^u = N^dk_\tau,\\
%    2 N_d + 2 N_u kt == P
    2N^d + 2N^uk_\tau = P,\\
%    2 N_u h + P y_c == 2 N_d l
    2hN^u + Py_c = 2lN^d.\\
      \end{alignedat}
  \right.
\end{equation}

В системе \ref{eq:case1_final_system} из трех уравнений содержится три неизвестные - $N^u, N^d, k_\tau$.




Для неизвестной $k_\tau$ доступны два варианта решения:

\begin{equation}
\label{eq:case1_k12}
  \begin{multlined}
    k_1 = -\dfrac{h + \sqrt{h^2 - 4y_c^2 + 4ly_c}}{2y_c} \\
    k_2 = -\dfrac{h - \sqrt{h^2 - 4y_c^2 + 4ly_c}}{2y_c} \\
  \end{multlined}
\end{equation}


Выберем тот, который удовлетворяет условию $k_\tau > 0$. По условию задачи, имеем $0 < y_c < l$ и $h > 0$. Проверим $k_1$:

\[
\begin{alignedat}{3}
  k_1 = - \dfrac{h - \sqrt{h^2 - 4y_c^2+4ly_c}}{2y_c} &\vee 0 \\
  \sqrt{h^2 - 4y_c^2+4ly_c} &\vee h\\
  h^2 - 4y_c^2+4ly_c &\vee h^2 \\
  4ly_c &\vee 4y_c^2 \\
  l &\vee y_c\\
  l &> y_c
\end{alignedat}
\]

Значит, $k_1 > 0$ при $0 < y_c < l$ и $h > 0$, что полностью соответствует условию \todo{вставить ссылку на условия} задачи.



Теперь проверим $k_2$. По условию задачи имеем $h > 0$ и $y_c > 0$, поэтому для $k_2$ всегда верно:
\[
    k_2 = -\dfrac{h + \sqrt{h^2 - 4y_c^2 + 4ly_c}}{2y_c} < 0\\
\]

Значение \ref{eq:case1_k12} $k_2$ не подходит по условию задачи.

Было установлено при каких условиях $k_1$ из решения \ref{eq:case1_k12} принимает значения больше нуля. Теперь установим, при каких значениях параметров $h,l$ и $y_c$ значения $k_1$ меньше единицы. Значения, большие единицы, для коэффициента трения означают, что веса аппарата не хватает чтобы создать нужную силу трения и требуется использовать специальные зацепные механизмы на концах ног чтобы обеспечить устойчивость конфигурации.

Решим следующее неравенство в переменных $h,l,y_c$:
\begin{equation}
\label{eq:case1_k_less_than_1}
  \begin{alignedat}{3}
  k_1 = - \dfrac{h - \sqrt{h^2 - 4y_c^2+4ly_c}}{2y_c} & < 1 \\
  \end{alignedat}
\end{equation}


%старый текст идет ниже
Найдём моменты реакций опоры и силы тяжести:

Аналогично находим остальные моменты:

$$
mom_{\overline{r}_2}\overline{R}_2 = N_2(-\zeta_F,k_\zeta^2\xi_F+k_\xi^2\zeta_F,-\xi_F)
$$

$$
mom_{\overline{r}_3}\overline{R}_3 = N_3(\eta_R,-\xi_R,k_\eta^3\xi_R-k_\xi^3\eta_R)
$$

$$
mom_{\overline{r}_4}\overline{R}_4 = N_4(\eta_R,\xi_R,-k_\eta^4\xi_R-k_\xi^4\eta_R)
$$

$$
mom_{\overline{r}_c}\overline{P} = P(-\eta_c,\xi_c,0)
$$

Сумма моментов внешних сил, действующих на систему должна быть равна нулю:

$$
\sum_{i=1}^4 mom_{\overline{r}_i}\overline{R}_i+mom_{\overline{r}_c}\overline{P} = 0
$$

В скалярном виде уравнения для моментов записываются в следующем виде:

\begin{equation}
\left\{
\begin{array}{rcl}
-N_1\zeta_F-N_2\zeta_F+N_3\eta_R+N_4\eta_R & = & P\eta_c,\\
\\
N_1(-k_\zeta^1\xi_F+k_\xi^1\zeta_F)+N_2(k_\zeta^2\xi_F+k_\xi^2\zeta_F)-N_3\xi_R+N_4\xi_R & = & -P\xi_c,\\
\\
N_1\xi_F-N_2\xi_F+N_3(k_\eta^3\xi_R-k_\xi^3\eta_R)+N_4(-k_\eta^4\xi_R-k_\xi^4\eta_R) & = & 0.\\
\end{array}
\right.
\end{equation}

Уравнения статики содержат 6 уравнений и двенадцать неизвестных. Введём дополнительные соотношения между неизвестными для уменьшения числа независимых неизвестных. Из соображений симметрии введём следующие соотношения:

\begin{equation}
\begin{array}{cc}
N_F := N_1 = N_2, & N_R := N_3 = N_4,\\
\\
k_\xi^F = k_\xi^2 = -k_\xi^2, & k_\xi^R = k_\xi^3 = -k_\xi^4,\\
\\
k_\zeta^F = k_\zeta^1 = k_\zeta^2, & k_\eta^R = k_\eta^3 = k_\eta^4.\\
\end{array}
\end{equation}

После подстановки в систему (\textbf{вставить ссылку}) первое уравнение превращается в тождество, второе и третье уравнения преобразуются к следующему виду:

\begin{equation}
\left\{
\begin{array}{rcl}
 N_F+N_Rk_\eta^R & = & 0,\\
 \\
 N_Fk_\zeta^F+N_R & = & \dfrac{P}{2}.\\
\end{array}
\right.
\end{equation}

В системе (\textbf{вставить ссылку}) после подстановки, третье уравнение выраждается в тождество, а из второго уравнения следует, что

$$
\xi_c \equiv 0
$$

Оставшееся первое уравнение реобразовывается к виду:

$$
N_R\eta_R-N_F\zeta_F = \dfrac{P\eta_c}{2}
$$

В итоге, получается следующая система из трёх уравнений с четыремя неизвестными:

\begin{equation}
\left\{
\begin{array}{rcl}
 N_F+N_Rk_\eta^R & = & 0\\
 \\
 N_Fk_\zeta^F+N_R & = & \dfrac{P}{2}\\
 \\
 N_R\eta_R-N_F\zeta_F & = & \dfrac{P\eta_c}{2} \\
\end{array}
\right.
\end{equation}

В полученной системе три уравнения и четыре неизвестных. Требуется дополнительное соотношение для замыкания системы. Свяжем коэффициенты $k_\zeta^F$ и $k_\eta^R$ следующим соотношением:

\begin{equation}
k = k_\zeta^F = -k_\eta^R
\end{equation}

Выбор разных знаков для $k_\zeta^F$ и $k_\eta^R$ не случаен. Из первого уравнения системы (\textbf{вставить ссылку}) и условия $N_R,N_F \ge 0$ следует

$$
k_\eta^R = -\dfrac{N_F}{N_R} \leq 0
$$


Введённая связь коэффициентов трения указывает на требование более равномерного распределения силы трения между передними и задними ногами. Равномерное распределение сил в касательных плоскостях к опорным поверхностям между передней и задней парой ног влечёт меньшую нагрузку на шарниры ног, отвечающие за поворот плоскости ноги относительно сегментов корпуса.

После подстановки, получаем:

\begin{equation}
\left\{
\begin{array}{rcl}
    N_F-N_Rk &=& 0\\
    \\
    N_Fk+N_R &=& \dfrac{p}{2}\\
    \\
    N_R\eta_R-N_F\zeta_F &=& \dfrac{P\eta_c}{2}\\
\end{array}
\right.
\end{equation}

Найдем неизвестные $N_R, N_F, k$. Из первого уравнения получаем:

$$
N_F = N_Rk
$$

Подставляя во второе уравнение, получаем:

$$
N_R(k^2+1) = \dfrac{P}{2} \Rightarrow N_R = \dfrac{P}{2(1+k^2)}
$$

Подставляя варжения для $N_R$ и $N_F$ в третье уравнение системы, получим квадратное уравнение относительно $k$:

\begin{equation}
\eta_ck^2+\zeta_Fk+(\eta_c-\eta_R) = 0
\end{equation}

Дискриминант уравнения всегда больше нуля при $\eta_R-\eta_c\geq0$ (\textbf{? - ведь может быть и так, что $\eta_R-\eta_c < 0$, $а D>0$}):

$$
D = \zeta_F^2+4\eta_c(\eta_R-\eta_c)
$$

Решения записываются в виде:

\begin{equation}
k_{1,2} = \dfrac{-\zeta_F\pm\sqrt{\zeta_F^2+4\eta_c(\eta_R-\eta_c)}}{2\eta_c}
\end{equation}

Нам подходит только $k>0$ (\textbf{а в этом случае $k$ всегда $>0$? см. предыдущее замечание: $\eta_R>\eta_c$?}):

$$
k = \dfrac{-\zeta_F+\sqrt{\zeta_F^2+4\eta_c(\eta_R-\eta_c)}}{2\eta_c}
$$

Важно знать при каких условиях коэффициент $k$ (\textbf{нигде раньше не было явно сказано, что это коэфф. трения, об это надо сказать раньше, вообще про все коэффициенты надо сразу все объявлять, у тебя не везде это есть}) по своему модулю значения не превышает единицу. Значение $k$ большее единицы означает что аппарат не сможет сохранять равновесие за счёт силы трения и ему потребуются специальные зацепные устройства для обеспечения коэффициента трения больше единицы.

Найдём при каких значениях параметров коэффициент трения $k$ по модулю меньше единицы. После всех преобразований получаем неравенство:

\begin{equation}
\eta_R-2\eta_c-\zeta_F < 0
\end{equation}

В пространстве параметров $(\eta_R,\eta_c\zeta_F)$ (\textbf{-здесь пропущена запятая?}) область решения лежит в квадранте $\eta_r\geq0, \eta_c\geq0,\zeta_F\geq0$ и ограниченна плоскостью

$$
\eta_R-2\eta_c-\zeta_F = 0
$$

Полученное неравенство можно интерпретировать следующим образом. Если центр масс аппарата находится на расстоянии от вертикальной площадки более чем на $\dfrac{\eta_c}{2}$, то ноги, опирающиеся на вертикальную площадку, могут быть выставлены на любую высоту и силы трения хватит для сохранения равновесия. Если подвести центр масс аппарата к вертикальной стенке на расстояние менее $\dfrac{\eta_c}{2}$, то появится ограничение на минимальную высоту установки передних ног $\zeta_F$, при которой аппарат сможет сохранить равновесие при $k<1$.

Подставляя полученное выражение для $k$ в выражения для $N_R$ и $N_F$ находим распределение нормальных составлящих реакций опоры.

$$
N_R = \dfrac{P}{2(1+k^2)}, N_F = N_Rk = \dfrac{Pk}{2(1+k^2)}
$$

\subsection{второй случай}

Рассмотрим ситуацию когда аппарат опирается на верхнюю и вертикальную площадку. (\textbf{непонятно как})

вставить рисунок


Пусть точки контакта ног с опорными площадками:

$$
\begin{array}{ll}
\overline{r}_1 = (\xi_F,-\eta_F,h) & \overline{r}_2 = (-\xi_F,-\eta_F,h)\\
\overline{r}_3 = (\xi_R,0,\zeta_r) & \overline{r}_4 = (-\xi_R,0,\zeta_R)\\
\end{array}
$$

Положение центра масс аппарата и вес аппарата:

$$
\begin{array}{cc}
\overline{r}_c = (\xi_c,\eta_c,\zeta_c) & \overline{P} = (0,0,-P = -Mg)\\
\end{array}
$$

Реакции в опорных точках:

$$
\begin{array}{ll}
\overline{R}_1 = N_1(k_\xi^1,k_\eta^1,1) & \overline{R}_2 = N_2(k_\xi^2,k_\eta^2,1)\\
\\
\overline{R}_3 = N_3(k_\xi^3,1,k_\zeta^3) & \overline{R}_4 = N_4(k_\xi^4,1,k_\zeta^4)\\
\end{array}
$$

первая система уравнений статики записывается в следующем виде:

\begin{equation}
\left\{
\begin{array}{rcl}
    N_1k_\xi^1+N_2k_\xi^2+N_3k_\xi^3+N_4k_\xi^4 & = & 0\\
    \\
    N_1k_\eta^1+N_2k_\eta^2+N_3+N_4 & = & 0\\
    \\
    N_1+N_2+N_3k_\zeta^3+N_4k_\zeta^4 & = & P\\
\end{array}
\right.
\end{equation}

Вычислим моменты реакций опоры и момент силы тяжести:

$$
mom_{\overline{r}_1}\overline{R}_1 = 
\left|
\begin{array}{ccc}
\overline{i} & \overline{j} & \overline{k}\\
\xi_F & -\eta_F & h\\
N_1k_\xi^1 & N_1k_\eta^1 & N_1\\
\end{array}
\right| = N_1(-\eta_F-k_\eta^1h,-\xi_F+k_\xi^1h,k_\eta^1\xi_F+k_\xi^1\eta_F)
$$

Аналогично находим остальные моменты:

$$
mom_{\overline{r}_2}\overline{R}_2 = N_2(-\eta_F-k_\eta^2h,\xi_F+k_\xi^2h,-k_\eta^2\xi_F+k_\xi^2\eta_F)
$$

$$
mom_{\overline{r}_3}\overline{R}_3 = N_3(-\zeta_R,-k_\zeta^3\xi_R+k_\xi^3\zeta_R,\xi_R)
$$

$$
mom_{\overline{r}_4}\overline{R}_4 = N_4(-\zeta_R,k_\zeta^4\xi_R+k_\xi^4\zeta_R,-\xi_R)
$$

$$
mom_{\overline{r}_c}\overline{P} = P(-\eta_c,\xi_c,0)
$$

Система уравнений для суммы моментов внешних сил записывается в следующем виде:

\begin{equation}
\left\{
\begin{array}{rcl}
-N_1(\eta_F+k_\eta^1h)-N_2(\eta_F+k_\eta^2)-N_3\zeta_R-N_4\zeta_R & = & P\eta_c\\
\\
N_1(-\xi_F+k_\xi^1h)+N_2(\xi_F+k_\xi^2h)+\\
\\
+N_3(-k_\zeta^3\xi_R+k_\xi^3\zeta_R)+N_4(k_\zeta^4\xi_R+k_\xi^4\zeta_R) & = & -P\xi_c\\
\\
N_1(k_\eta^1\xi_f+k_\xi^1\eta_F)+N_2(-k_\eta^2\xi_F+k_\xi^2\eta_F)+N_3\xi_R-N_4\xi_R & = & 0\\
\end{array}
\right.
\end{equation}

Получилась система из 6 уравнений и 12 неизвестных. Из соображений симметрии введём дополнительные определяющие соотношения:

$$
\begin{array}{ll}
N_F:=N_1=N_2 & N_R:=N_3=N_4 \\
\\
k_\xi^F:= k_\xi^1 = -k_\xi^2 & k_\xi^R:=k_\xi^3 = -k_\xi^4 \\
\\
k_\zeta^R:=k_\zeta^3 = k_\zeta^4 & k_\eta^F:=k_\eta^1 = k_\eta^2 \\
\end{array}
$$

После всех преобразований получим систему из трёх уравнений с четыремя неизвестными:

\begin{equation}
\left\{
\begin{array}{rcl}
N_Fk_\eta^F+N_R & = & 0\\
\\
N_F+N_Rk_\zeta^R & = & \dfrac{P}{2}\\
\\
N_F(\eta_F+k_\eta^Fh)+N_R\zeta_R & = & -\dfrac{P\eta_c}{2}\\
\end{array}
\right.
\end{equation}

Введём дополнительное соотношение:

$$
k:=k_\zeta^R = -k_\eta^F>0
$$

После подстановки в систему (\textbf{вставить ссылку}) получаем:

\begin{equation}
\left\{
\begin{array}{rcl}
N_R-N_Fk & = & 0\\
\\
N_F+N_Rk & = & \dfrac{P}{2}\\
\\
N_F(\eta_F-kh)+N_R\zeta_R & = & -\dfrac{P\eta_c}{2}\\
\end{array}
\right.
\end{equation}

Найдём неизвестные $N_R,N_F,k$.

Из первого уравнения получаем:

$$
N_R = N_Fk
$$

После подстановки во второе уравнение находим:

$$
N_F = \dfrac{P}{2(1+k^2)}
$$

Подставляем всё в третье уравнение и получаем квадратное уравнение относительно $k$:

\begin{equation}
\eta_ck^2-k(h-\zeta_R)+(\eta_c+\eta_F) = 0
\end{equation}

Дискриминант уравнения:

$$
D = (h-\zeta_R)^2-4\eta_c(\eta_c+\eta_F)
$$

Для существования решений необходимо чтобы дискриминант $D$ был больше, либо равен нулю.

Решим неравенство $(h-\zeta_R)^2-4\eta_c(\eta_c+\eta_F) \geq 0$. Введём новые переменные:

$$
x:=\eta_F,y := \eta_c, z:= (h-\zeta_R)
$$

В новых координатах неравенство запишется в виде:

$$
z^2-4y(y+x) \geq 0 \Leftrightarrow 
\left[
\begin{array}{rl}
z\geq2\sqrt{y(y+x)}, & z\geq 0  \\
z\leq-2\sqrt{y(y+x)}, & z \leq 0 \\
\end{array}
\right.,
y(y+x)\geq 0
$$

Область допустимых значений для дискриминанта лежит между парой пересекающихся плоскостей $y(y+x) \geq 0$ без внутренностей эллиптического конуса $z^2-4y(y+x)\geq 0$.

\textbf{Вставить картинку ОДЗ}

Корни уравнения (\textbf{вставить ссылку}) могут принимать как отрицательные значения, так и положительные. Нельзя сразу сказать какой корень нам подходит.

$$
k_{1,2} = \dfrac{(h-\zeta_R)\pm\sqrt{(h-\zeta_R)^2-4\eta_c(\eta_c+\eta_F)}}{2\eta_c}
$$

Исследуем при каких условиях полученные коэффициенты удовлетворяют условию $0<k_{1,2}<1$.

\subsection{$k_1$}

$$
k_1 = \dfrac{(h-\zeta_R)-\sqrt{(h-\zeta_R)^2-4\eta_c(\eta_c+\eta_F)}}{2\eta_c}
$$

Сделаем замену:

$$
x:=\eta_F,y := \eta_c, z:= (h-\zeta_R)
$$

Выражение для $k_1$ принимает вид

$$
k_1 = \dfrac{z-\sqrt{z^2-4y(y+x)}}{2y}
$$

Неравенство $k_1>0$ эквивалентно совокупности:

$$
\left[
\begin{array}{l}
\left\{
    \begin{array}{l}
    z-\sqrt{z^2-4y(x+y)}>0\\
    y>0 \\
    \end{array}
    \right.\\
    \\
    \left\{
    \begin{array}{l}
    z-\sqrt{z^2-4y(x+y)}<0\\
    y<0\\
    \end{array}
    \right.\\
\end{array}
\right.
$$

Далее, будем всегда учитывать ОДЗ определённую выше. Рассмотрим первую систему. Для первого неравенства верна следующая эквивалентность:

$$
z>\sqrt{z^2-4y(x+y)} \Leftrightarrow \left\{ \begin{array}{ccl}
0 & \leq & z^2-4y(x+y)  \\
z & > & 0 \\
z^2 & > & z^2-4y(x+y)\\
\end{array}
\right.
$$

После приведения подобных слагаемых получаем решение:

$$
\left\{
\begin{array}{ccl}
z & \geq & 2\sqrt{y(x+y)} \\
\\
0 & < & y(x+y) \\
\\
y & > & 0\\
\end{array}
\right.
$$

Полученное множество точек $(x,y,z)$ соответствует части ОДЗ где $y>0$ и $z>0$.

Рассмотрим вторую систему. Для первого неравенства верна следующая эквивалентность:

$$
z<\sqrt{z^2-4y(x+y)}\Leftrightarrow
\left[
\begin{array}{l}
    \left\{
    \begin{array}{l}
    z<0\\
    z^2-4y(x+y)\geq 0\\
    \end{array}
    \right.\\
    \\
    \left\{
    \begin{array}{l}
    z>0\\
    z^2<z^2-4y(x+y)\\
    \end{array}
    \right.\\
\end{array}
\right.
$$

Первая система имеет решение:

$$
\left\{
\begin{array}{l}
z<0\\
y<0\\
z\leq-2\sqrt{y(x+y)}\\
\end{array}
\right.
$$

Область решения совпадает с ОДЗ где $z<0$ и $y<0$.

Вторая система имеет решение:

$$
\left\{
\begin{array}{l}
    z>0\\
    y<0\\
    y(x+y)<0\\
\end{array}
\right.
$$

Полученная область решений для второй системы лежит вне области допустимых значений.

Подведём итог. Если провести координатные плоскости $z=0$ и $y=0$ то область допустимых значений условно делится на четыре части.
Решением неравенства $k_1>0$ будут квадратны где ${z>0,y>0}$ и ${z<0,y<0}$. Квадранты расположены по диагонали друг относительно друга.

Определим при каких $(x,y,z)$ выполняется условие $k_1<1$, т.е. при каком расположении точек опоры и центра масс системы аппарат сможет держаться на препятствии только за счёт трения. Если $k_1>0$ то аппарату нужны специальные зацепляющие устройства на подобии крючьев, присосок и т.п.

$$
\dfrac{z-\sqrt{z^2-4y(x+y)}}{2y}<1 \Leftrightarrow \dfrac{(z-2y)-\sqrt{z^2-4y(x+y)}}{2y}<0
$$

Полученное неравенство эквиваленто совокупности двух систем:

$$
\left[
\begin{array}{l}
    \left\{
    \begin{array}{l}
    (z-2y)-\sqrt{z^2-4y(x+y)}>0\\
    y<0\\
    \end{array}
    \right.\\
    \\
    \left\{
    \begin{array}{l}
    (z-2y)-\sqrt{z^2-4y(x+y)}<0\\
    y>0\\
    \end{array}
    \right.\\
\end{array}
\right.
$$

Рассмотрим первую систему. Для первого уравнения неравенства верна эквивалентность:

$$
(z-2y)>\sqrt{z^2-4y(x+y)} \Leftrightarrow
\left\{
    \begin{array}{l}
    z-2y>0\\
    (z-2y)^2>z^2-4y(x+y)\\
    \end{array}
\right.
$$

Решение первой системы записывается в следующем виде:

$$
\left\{
\begin{array}{l}
y<0\\
z>2y\\
z<2y+x\\
z<-2\sqrt{y(x+y)}\\
\end{array}
\right.
$$

Полученная область не пустая, заключена между плоскостями $z=2y+x$ и $z=2y$ и располгается в части ОДЗ где $x>0, y<0$ и $z<0$.

Рассмотрим вторую систему. Для первого уравнения получаем следующую эквивалентность:

$$
(z-2y)<\sqrt{z^2-4y(x+y)}\Leftrightarrow
\left[
\begin{array}{l}
    \left\{
        \begin{array}{l}
        z-2y<0\\
        z^2-4y(x+y)\geq0\\
        \end{array}
    \right.\\
    \\
    \left\{
        \begin{array}{l}
        z-2y>0\\
        (z-2y)^2<z^2-4y(x+y)\\
        \end{array}
    \right.\\
\end{array}
\right.
$$

Первая система совокупности имеет непустое решение:

$$
\left\{
\begin{array}{l}
 y>0\\
 z<2y\\
 z^2-4y(x+y)\geq0\\
\end{array}
\right.
$$

Вторая система так же имеет непустое решение:

$$
\left\{
\begin{array}{l}
y>0\\
z>2y\\
z>2y+x\\
\end{array}
\right.
$$

Решением исходного неравенства $0<k_1<1$ будет пересечение найденных областей.

\textbf{Вставить картинку.}


\subsection{$k_2$}

Решим неравенство $0<k_2<1$. Область допустимых значений совпадает аналогичными значениями для $k_1$. Начнём с решения неравенства $k_2>0$.

$$
k_2 = \dfrac{(h-\zeta_R)+\sqrt{(h-\zeta_R)^2-4\eta_c(\eta_c+\eta_F)}}{2\eta_c}>0
$$

После замены переменных:

$$
\dfrac{z+\sqrt{z^2-4y(x+y)}}{2y}>0
$$

Исходное неравенство эквивалентно совокупности двух систем:

$$
\left[
\begin{array}{l}
    \left\{
    \begin{array}{l}
     z+\sqrt{z^2-4y(x+y)}>0\\
     y>0\\
    \end{array}
    \right.\\
    \\
    \left\{
    \begin{array}{l}
     z+\sqrt{z^2-4y(x+y)}<0\\
     y<0\\
    \end{array}
    \right.\\
\end{array}
\right.
$$

Рассмотрим первую систему. Первое неравенство эквивалентно совокупности двух систем:

$$
z+\sqrt{z^2-4y(x+y)}>0 \Leftrightarrow 
\left[
    \begin{array}{l}
        \left\{
            \begin{array}{l}
             z>0\\
             z^2-4y(x+y)\geq 0\\
            \end{array}
        \right.\\
        \\
        \left\{
            \begin{array}{l}
             z<0\\
             z^2<z^2-4y(x+y)\\
            \end{array}
        \right.\\
    \end{array}
\right.
$$

Решением первой системы является четверь области допустимых значений, где $z>0$ и $y>0$.

Множеством решений второй системы является пустое множество.

Рассмотрим вторую систему совокупности для исходного неравенства. Для первого уравнения системы верна эквивалентность:

$$
z<-\sqrt{z^2-4y(x+y)} \Leftrightarrow
\left\{
\begin{array}{l}
z<0\\
z^2>z^2-4y(x+y)\\
\end{array}
\right.
$$

Решение записывается в виде:

$$
\left\{
\begin{array}{l}
 z<0\\
 y<0\\
 y(x+y)>0\\
\end{array}
\right.
$$

Полученная система соответствует одной из четвертей ОДЗ.

Решим неравенство $k_2<1$.

$$
\dfrac{z+\sqrt{z^2-4y(x+y)}}{2y}<1 \Leftrightarrow \dfrac{(z-2y)+\sqrt{z^2-4y(x+y)}}{2y}
$$

Получаем совокупность систем:

$$
\left[
\begin{array}{l}
    \left\{
        \begin{array}{l}
         (z-2y)<-\sqrt{z^2-4y(x+y)}\\
         y>0\\
        \end{array}
    \right.\\
    \\
    \left\{
        \begin{array}{l}
         (z-2y)>-\sqrt{z^2-4y(x+y)}\\
         y<0\\
        \end{array}
    \right.\\
\end{array}
\right.
$$

Первая система даёт непустое множество:

$$
\left\{
\begin{array}{l}
 z<2y\\
 z<2y+x\\
 y>0\\
\end{array}
\right.
$$

Втарая система преобразуется в совокупность:

$$
\left[
\begin{array}{l}
    \left\{
        \begin{array}{l}
         z>2y\\
         z^2-4y(x+y)\geq 0\\
         y<0\\
        \end{array}
    \right.\\
    \\
    \left\{
        \begin{array}{l}
         z<2y\\
         z>2y+x\\
         y<0\\
        \end{array}
    \right.\\
\end{array}
\right.
$$

Обе системы имеют непустые множества решений.

Получены решения для неравенств $0<k_{1,2}<0$ (\textbf{$<1$ ? :)}). Построим на графике области решений для $k_1$ и $k_2$.
Есть ли такие области в пространстве $(x,y,z)$ где действуют одновременно два коэффициента $k_1$ и $k_2$? Не стоит забывать, что нас интересует только область где $z>0$, т.к. $z$ по определению равно величине разницы $(h-\zeta_R)$. Нельзя поставить задние ноги на вертикальную стенку выше верхней площадки. Параметр $y:=\eta_c$ могжет принимать как отрицательные, так и положительные значения. Можно смещать центр масс аппарата в ту или иную сторону от кромки верхней площадки. Параметр $x:=\eta_F$ может принимать только положительные значения в силу того, что символ $\eta_f$ обозначает положительное число и нельзя ставить ноги на воздух вне верхней площадки препятствия.

Задача проверки статической устойчивости шестиногого шагающего аппарата с двенадцатью неизвестными и шестью уравнениями при помощи соображений симметрии свелась к системе из трёх уравнений с тремя неизвестными.

\textbf{Вставить картинку.}


Рассмотренные случаи постановки ног покрывают почти все стадии преодоления препятствия типа уступ. Остаётся рассмотреть фазу, когда робот поочередно переставляет задние ноги для смещения корпуса к вертикальной стенке, в то время как передняя пара и средняя пара уже опираются на вертикальную контактную площадку.

\textbf{Во-первых, в последних двух пунктах не хватает нумераций формул и ссылок на них, сложно читается без этого.
Во-вторых, во всех системах обычно ставят , и . я в начале начал править, но не знаю продолжать ли это. 
В-третьих, не понятен выбор названий для подсекций: первый этап, второй случай и трехопорная фаза - это все классификация схожих фаз? Это относится к 4 возможным ситуациям, про которые говорилось в начале? Если да, то опять же, их нужно подробно объяснить, может быть в начале каждой подсекции. И почему тогда трехопорная фаза - последний случай, ведь было рассмотрено только два перед этим. Или $k_1$ и $k_2$ - это тоже разные случаи? Тогда и это надо прямо указать. Вообще, подсекции $k_1$ и $k_2$ я бы сделал подподсекциями.}