
\chapter{Многоногий шагающий аппарат с трёхзвенный корпусом}
\section{Описание трехзвенного робота}

Рассмотрим робот с трехзвенным корпусом. Звенья корпуса соединены между собой при помощи одностепенных вращательных шарниров. 

\todo{картинка кинематической схемы робота}


\subsection{Преодоление препятствия типа уступ. Постановка задачи}

\todo{Вставить картинку}

Препятствие типа "Прямой уступ" состоит из трёх контактных площадок - две горизонтальные площадки и одна вертикальная. Расстояние по вертикали между двумя горизонтальными площадками $h$. В начальный момент времени робот стоит на нижней площадке, на некотором расстоянии от вертикальной стенки. Вертикальная стенка образует прямой угол с горизонтальными площадками.

Считаем, что робот двигается в режиме статической устойчивости и не менее трех ног находятся в контакте с опорной поверхностью. Работ двигается, так называемой, походкой галоп - поочередно переставляя пару симметричных ноги по левому и правому борту. Робот двигается достаточно медленно.

При залезании на верхнюю горизонтальную площадку прямого уступам можно выделить следующие конфигурации расположения ног робота:

\begin{enumerate}
  \item все ноги робота опираются на нижнюю горизонтальную площадку
  \item одна пара ног опирается на вертикальную площадку, а другая пара ног опирается на нижнюю горизонтальную площадку
  \item одна пара ног опирается на верхнюю контактную площадку, а другая пара ног опирается на вертикальную площадку
  \item все ноги робота опираются на верхнюю горизонтальную площадку уступа
\end{enumerate}


%Первый этап --- подход аппарата к вертикальной площадке и зацепление передними ногами за верхнюю кромку препятствия. В конце первого этапа корпус аппарата принимает вертикальное положение, передние ноги находятся в контакте с верхней площадкой, остальные ноги опираются на вертикальную площадку. Второй этап --- последовательное подтягивание корпуса и перемещение сегментов через верхнюю кромку препятствия. В конце второго этапа корпус робота принимает горизонтальное положение, все ноги аппарата находятся в контакте с верхней горизонтальной площадкой.
%
%Аппарат перемещается при помощи походки "галоп", т.е. переносятся симметричные пары ног --- две передние ноги, две средние, две задние. Будем считать, что аппарат двигается достаточно медленно. 

Исследуем статическую устойчивость робота для каждой конфигурации расположения ног на опорных площадках препятствия. 


\subsection{Первая конфигурация}
\todo{сделать ссылку на эту главу}
\todo{сделать нижние индексы и назвать их front и rare}

Исследуем на устойчивость конфигурацию, когда одна пара ног робота опирается на вертикальную стенку, а другая пара ног опирается на нижнюю горизонтальную площадку. Робот будет находится в статическом равновесии, когда сумма всех внешних сил и сумма всех внешних моментов будут равны нулю.

Введём неподвижную систему координат $Oxyz$. Оси системы направлены как показано на рис.(\todo{вставить ссылку}).

Пусть точки опоры ног имеют координаты:

\begin{equation}
  \label{eq:step_points}
  \begin{alignedat}{3}
    &\overline{r}_1 &= (d, 0, h) , &\overline{r}_2 &= (-d, 0, h) \\
    &\overline{r}_3 &= (d, l, 0) , &\overline{r}_4 &= (-d, l, 0) \\
  \end{alignedat}
\end{equation}


Положение центра масс аппарата задано вектором:

\[\overline{r}_c = (0, y_c, z_c)\]

На центр масс робота действует сила тяжести:

\begin{equation}
\label{eq:F_c}
  \overline{P} = (0,0,-P)
\end{equation}

В точках контакта ног с опорной поверхностью, на ноги действуют реакции опоры:

\begin{equation}
\label{eq:reactions}
  \overline{R}_i := N^i\cdot\overline{n}_i + F_\tau^i\cdot\overline{\tau}_i + F_\nu^i\cdot\overline{\nu}_i = \overline{N}^i + \overline{F}_\tau^i + \overline{F}_\nu^i,
\end{equation}

где $N_i$ - модуль нормальной составляющей $i$-й реакции, направленный по нормали $\overline{n}_i$, $F_\tau^i$ и $F_\nu^i$ - проекции касательной составляющей реакции, в проекции на локальные оси $\overline{\tau}_i$ и $\overline{\nu}_i$.

Чтобы система находилась в положении равновесия сумма всех внешних сил должна быть равна нулю, а также сумма всех моментов внешних сил относительно любой точки должна быть равна нулю:

\begin{equation}
\label{eq:general_static_equations}
  \left\{
    \begin{alignedat}{3}
      &\sum_{i=1}^4\overline{R}_i+\overline{P} = 0, \\
      &\sum_{i=1}^4[\overline{r}_i\times\overline{R}_i] + [\overline{r}_c\times\overline{P}] = 0\\
    \end{alignedat}
  \right.
\end{equation}



Распишем все слагаемые из уравнения \ref{eq:reactions}.

Нормальные составляющие $\overline{N}^i$ реакций опоры:

\begin{equation}
\label{eq:N_i}
  \begin{alignedat}{3}
    &\overline{N}^1 &= &N_1\cdot(0,1,0),  \\
    &\overline{N}^2 &= &N_2\cdot(0,1,0),  \\
    &\overline{N}^3 &= &N_3\cdot(0,0,1),  \\
    &\overline{N}^4 &= &N_4\cdot(0,0,1).  \\
  \end{alignedat}
\end{equation}

Соответствующие силы трения в точках контакта:
\begin{equation}
\label{eq:F_tau_i}
  \begin{alignedat}{3}
    &\overline{F}_\tau^1 &= &F_\tau^1\cdot(0,0,1),  \\
    &\overline{F}_\tau^2 &= &F_\tau^2\cdot(0,0,1),  \\
    &\overline{F}_\tau^3 &= &F_\tau^3\cdot(0,1,0),  \\
    &\overline{F}_\tau^4 &= &F_\tau^4\cdot(0,1,0).  \\
  \end{alignedat}
\end{equation}

\begin{equation}
\label{eq:F_nu_i}
  \begin{alignedat}{1}
  \overline{F}_\nu^1 = F_\nu^1\cdot(1,0,0), \\
  \overline{F}_\nu^2 = F_\nu^2\cdot(1,0,0), \\
  \overline{F}_\nu^3 = F_\nu^3\cdot(1,0,0), \\
  \overline{F}_\nu^4 = F_\nu^4\cdot(1,0,0). \\
  \end{alignedat}
\end{equation}

Рассчитаем моменты сил для реакций опоры и силы тяжести относительно начала системы координат $Oxyz$.

\begin{equation}
\label{eq:momentums}
  \begin{alignedat}{3}
    &[\overline{r}_1\times\overline{R}_1] &=&  (d,0,h)\times(F_\nu^1, N^1, F_\tau^1) &=& (-N^1h,F_\nu^1h - F_\tau^1d,N^1d),  \\
    &[\overline{r}_2\times\overline{R}_2] &=& (-d,0,h)\times(F_\nu^2, N^2, F_\tau^2) &=& (-N^2h,F_\nu^2h + F_\tau^2d,-N^2d), \\
    &[\overline{r}_3\times\overline{R}_3] &=&  (d,l,0)\times(F_\nu^3, F_\tau^3, N^3) &=& (N^3l,-N^3d, F_\tau^3d-F_\nu^3l), \\
    &[\overline{r}_4\times\overline{R}_4] &=& (-d,l,0)\times(F_\nu^4, F_\tau^4, N^4) &=& (N^4l, N^4d,-F_\tau^4d-F_\nu^4l),\\
    &[\overline{r}_c\times\overline{P}]   &=&  (0, y_c, z_c)\times(0,0,-P) &=& (-Py_c,0,0)\\
  \end{alignedat}
\end{equation}

Подставим полученные выражения для сил \ref{eq:N_i}, \ref{eq:F_nu_i}, \ref{eq:F_tau_i}, \ref{eq:F_c} и моментов \ref{eq:momentums} в уравнения статического равновесия (\ref{eq:general_static_equations}) и получим следующую систему из шести уравнений:

\begin{equation}
  \label{eq:case1_initial}
  \left\{
    \begin{alignedat}{3}  
      %Fnu_1 + Fnu_2 + Fnu_3 + Fnu_4 == 0
      %Ft_3 + Ft_4 + N_1 + N_2 == 0
      %Ft_1 + Ft_2 + N_3 + N_4 == P  
      F_\nu^1 + F_\nu^2 + F_\nu^3 + F_\nu^4 = 0, \\
      N^1 + N^2 + F_\tau^3 + F_\tau^4 = 0, \\
      F_\tau^1 + F_\tau^2 + N^3 + N^4 = P, \\
      %  N_1 h + N_2 h + P y_c == l (N_3 + N_4)
      %  d (Ft_1 + N_3) == Ft_2 d + Fnu_1 h + Fnu_2 h + N_4 d  
      %  d (Ft_3 + N_1) == Ft_4 d + N_2 d + Fnu_3 l + Fnu_4 l  
      N^1h + N^2h + Py_c = l(N^3 + N^4), \\
      d(F_\tau^1 + N^3) = F_\tau^2d + F_\nu^1h + F_\nu^2h + N^4d, \\
      d(F_\tau^3 + N^1) = F_\tau^4d + N^2d +F_\nu^3l + F_\nu^4l.\\
    \end{alignedat}
  \right.
\end{equation}



Используем модель кулоновского трения и выполним замены $F_\tau^i = k_\tau^i\cdot N^i$ и $F_\nu^i = k_\nu^i\cdot N^i$ в уравнении \ref{eq:case1_initial}. Получим следующий вид уравнений:

\begin{equation}
\label{eq:case1_subs_kulon}
\left\{
  \begin{alignedat}{3}  
    %N_1 knu_1 + N_2 knu_2 + N_3 knu_3 + N_4 knu_4 == 0
    N^1k_\nu^1 + N^2k_\nu^2 + N^3k_\nu^3 + N^4k_\nu^4 = 0,\\
    %N_1 + N_2 + N_3 kt_3 + N_4 kt_4 == 0
    N^1 + N^2 + N^3k_\tau^3 + N^4k_\tau^4 = 0, \\
    %N_3 + N_4 + N_1 kt_1 + N_2 kt_2 == P
    N^3 + N^4 + N^1k_\tau^1 + N^2k_\tau^2 = P, \\
    %N_1 h + N_2 h + P y_c == l (N_3 + N_4)
    N^1h + N^2h + Py_c = l(N^3 + N^4), \\
    %d (N_3 + N_1 kt_1) == N_4 d + N_2 d kt_2 + N_1 h knu_1 + N_2 h knu_2
    d(N^3 + N^1k_\tau^1) = dN^4 + dN^2k_\tau^2 + hN^1k_\nu^1 + hN^2k_\nu^2,\\
    %d (N_1 + N_3 kt_3) == N_2 d + N_4 d kt_4 + N_3 knu_3 l + N_4 knu_4 l
    d(N^1 + N^3k_\tau^3) = dN^2 + dN^4k_\tau^4 + lN^3k_\nu^3 + lN^4k_\nu^4\\
  \end{alignedat}
\right.
\end{equation}


В полученной системе \ref{eq:case1_subs_kulon} из шести уравнений содержится двенадцать неизвестных:
$k_\nu^1, k_\nu^2, k_\nu^3, k_\nu^4, k_\tau^1, k_\tau^2, k_\tau^3, k_\tau^4, N_1, N_2, N_3, N_4$ --- неизвестные, причём $N_i>0$. Введем дополнительное соотношение на коэффициенты трения:

\begin{equation}
  \label{eq:case1_assumption_1}
  \left\{
    \begin{alignedat}{3}
      k_\nu^1  &= -&k_\nu^2& &= k_\nu,\\
      k_\nu^3  &= -&k_\nu^4& &= k_\nu,\\
      k_\tau^1 &= &k_\tau^2& &= k_\tau^u,\\
      k_\tau^3 &= &k_\tau^4& &= k_\tau^d,\\
    \end{alignedat}
  \right.
\end{equation}

После подстановки дополнительных соотношений \ref{eq:case1_assumption_1} в \ref{eq:case1_subs_kulon} получим следующую систему уравнений:

\begin{equation}
\label{eq:case1_subs_assumption_1}
  \left\{
    \begin{alignedat}{3}
      k_\nu(N^1 + N^3 - N^2 - N^4) = 0,\\
      N^1 + N^2 + k_\tau^d(N^3 + N^4) = 0,\\
      k_\tau^u(N^1 + N^2) + N^3 + N^4 = P,\\
      N^1h + N^2h + Py_c = l(N^3 + N^4), \\      
      %N_3 d + N_1 d kt_u + N_2 h knu == N_4 d + N_2 d kt_u + N_1 h knu  
      dN^3 + dN^1k_\tau^u + hN^2k_\nu = dN^4 + dN^2k_\tau^u + hN^1k_\nu,\\
      %N_1 d + N_3 d kt_d + N_4 knu l == N_2 d + N_4 d kt_d + N_3 knu l
      dN^1 + dN^3k_\tau^d + lN^4k_\nu = dN^2 + dN^4k_\tau^d + lN^3k_\nu.\\
    \end{alignedat}
  \right.
\end{equation}

В системе \ref{eq:case1_subs_assumption_1} из шести уравнений уже семь неизвестных. Дополнительно введем предположение что правые и левые ноги одинаково нагружены, т.е.:

\begin{equation}
\label{eq:case1_assumption_2}
  \left\{
    \begin{alignedat}{3}
    N^1 = N^3 = N^u, \\
    N^2 = N^4 = N^d. \\
    \end{alignedat}
  \right.
\end{equation}


Подставим соотношение \ref{eq:case1_assumption_2} в систему \ref{eq:case1_subs_assumption_1}. В результате первое, пятое и шестое уравнение \ref{eq:case1_subs_assumption_1} превратятся в тождества:

\begin{equation}
\label{eq:case1_subs_assumption_2}
  \left\{
  \begin{alignedat}{3}
    % N_u + N_d kt_d == 0
    N^u + N^dk_\tau^d = 0, \\
    %2 N_d + 2 N_u kt_u == P
    2N^d + 2N^uk_\tau^u = P, \\
    %2 N_u h + P y_c == 2 N_d l
    2hN^u + Py_c = 2lN^d.\\
  \end{alignedat}
  \right.
\end{equation}

В системе \ref{eq:case1_subs_assumption_2} уже три уравнения и четыре неизвестных. Введем допущение, что коэффициенты трения для передних и задних ног совпадают: 

\begin{equation}
  \label{eq:case1_assumption_3}
  k_\tau^u = k_\tau^d = k_\tau > 0
\end{equation}


Подставив \ref{eq:case1_assumption_3} в \ref{eq:case1_subs_assumption_2} в итоге получим:

\begin{equation}
  \label{eq:case1_final_system}
  \left\{
  \begin{alignedat}{2}
%    N_u == N_d kt
    N^u = N^dk_\tau,\\
%    2 N_d + 2 N_u kt == P
    2N^d + 2N^uk_\tau = P,\\
%    2 N_u h + P y_c == 2 N_d l
    2hN^u + Py_c = 2lN^d.\\
      \end{alignedat}
  \right.
\end{equation}

В системе \ref{eq:case1_final_system} из трех уравнений содержится три неизвестные - $N^u, N^d, k_\tau$.

Из первого уравнения \ref{eq:case1_final_system} получаем:

\begin{equation}
\label{eq:case1_Nu}
  %N_u - N_d*k == 0;
  N^u = N^dk_\tau
\end{equation}

Подставляем выражение \ref{eq:case1_Nu} во второе уравнение \ref{eq:case1_final_system} и получаем:

\begin{equation}
  \label{eq:case1_Nd}
 % 2*N_d + 2*N_u*k == P;
  N^d = \dfrac{P}{2(1+k_\tau^2)}
\end{equation}


Далее, подставим \ref{eq:case1_Nd} и \ref{eq:case1_Nu} в третье уравнение \ref{eq:case1_final_system} и получим квадратное уравнение относительно $k_\tau$:

\begin{equation}
\label{eq:case1_k}
y_ck_\tau^2+hk_\tau + (y_c-l) = 0
\end{equation}

Для неизвестной $k_\tau$ доступны два варианта решения:

\begin{equation}
\label{eq:case1_k12}
  \begin{multlined}
    k_1 = -\dfrac{h + \sqrt{h^2 - 4y_c^2 + 4ly_c}}{2y_c} \\
    k_2 = -\dfrac{h - \sqrt{h^2 - 4y_c^2 + 4ly_c}}{2y_c} \\
  \end{multlined}
\end{equation}


Выберем тот, который удовлетворяет условию $k_\tau > 0$. По условию задачи, имеем $0 < y_c < l$ и $h > 0$. Проверим $k_1$:

\[
\begin{alignedat}{3}
  k_1 = - \dfrac{h - \sqrt{h^2 - 4y_c^2+4ly_c}}{2y_c} &\vee 0 \\
  \sqrt{h^2 - 4y_c^2+4ly_c} &\vee h\\
  h^2 - 4y_c^2+4ly_c &\vee h^2 \\
  4ly_c &\vee 4y_c^2 \\
  l &\vee y_c\\
  l &> y_c
\end{alignedat}
\]

Значит, $k_1 > 0$ при $0 < y_c < l$ и $h > 0$, что полностью соответствует условию \todo{вставить ссылку на условия} задачи.



Теперь проверим $k_2$. По условию задачи имеем $h > 0$ и $y_c > 0$, поэтому для $k_2$ всегда верно:
\[
    k_2 = -\dfrac{h + \sqrt{h^2 - 4y_c^2 + 4ly_c}}{2y_c} < 0\\
\]

Значение \ref{eq:case1_k12} $k_2$ не подходит по условию задачи.

Было установлено при каких условиях $k_1$ из решения \ref{eq:case1_k12} принимает значения больше нуля. Теперь установим, при каких значениях параметров $h,l$ и $y_c$ значения $k_1$ меньше единицы. Значения, большие единицы, для коэффициента трения означают, что веса аппарата не хватает чтобы создать нужную силу трения и требуется использовать специальные зацепные механизмы на концах ног чтобы обеспечить устойчивость конфигурации.

Решим следующее неравенство для переменной $y_c$:
\begin{equation}
\label{eq:case1_k_less_than_1}
  \begin{alignedat}{3}
  k_1 = - \dfrac{h - \sqrt{h^2 - 4y_c^2+4ly_c}}{2y_c} & \leq 1 \\
  -h + \sqrt(h^2 - 4y_c^2 + 4ly_c) &\leq 2y_c\\
  \sqrt{h^2 - 4y_c^2 + 4ly_c} &\leq 2y_c + h > 0\\
  h^2 - 4y_c^2 +4ly_c &\leq 4y_c^2 + 4hy_c + h^2\\
  l-y_c &\leq y_c + h\\
  y_c &\ge \dfrac{l-h}{2}
  \end{alignedat}
\end{equation}


Условие на $y_c$ из \ref{eq:case1_k_less_than_1} при заданных $l$ и $h$ используется во время управления роботом для проверки устойчивости конфигурации.


Чтобы найти реакции опоры, нужно подставить решение для $k_\tau$ \ref{eq:case1_k12} в уравнения \ref{eq:case1_Nd} и \ref{eq:case1_Nu}. 

Получены условия на положение центра масс робота при заданных опорных точках, при которых рассмотренная конфигурация будет устойчивой. Перейдем к рассмотрению следующей конфигурации.

Из уравнений \todo{ставить ссылку} следует, что необходимый коэффициент трения можно уменьшить за счет смещения центра масс ближе задним опорным ногам и за счет увеличения высоты $h$ опоры передних ног.


\subsection{Вторая конфигурация}

Рассмотрим конфигурацию, когда аппарат опирается на верхнюю и вертикальную площадку.

\todo{вставить рисунок}

Пусть теперь опорные точки ног имею координаты:

\begin{equation}
\label{eq:step_points_phase_2}
\begin{alignedat}{3}
&\overline{r}_1 &= (d, l, H) , &\overline{r}_2 &= (-d, l, H) \\
&\overline{r}_3 &= (d, 0, h) , &\overline{r}_4 &= (-d, 0, h) \\
\end{alignedat}
\end{equation}

Где $l < 0$ и $H > h$ и всегда выполняется $l < y_c$.

Нормальные составляющие реакций и силы трения соответственно:

\begin{equation}
\label{eq:N_i}
\begin{alignedat}{3}
&\overline{N}^1 &= &N_1\cdot(0,0,1),  \\
&\overline{N}^2 &= &N_2\cdot(0,0,1),  \\
&\overline{N}^3 &= &N_3\cdot(0,1,0),  \\
&\overline{N}^4 &= &N_4\cdot(0,1,0).  \\
\end{alignedat}
\end{equation}

Соответствующие силы трения в точках контакта:
\begin{equation}
\label{eq:F_tau_i}
\begin{alignedat}{3}
&\overline{F}_\tau^1 &= &F_\tau^1\cdot(0,1,0),  \\
&\overline{F}_\tau^2 &= &F_\tau^2\cdot(0,1,0),  \\
&\overline{F}_\tau^3 &= &F_\tau^3\cdot(0,0,1),  \\
&\overline{F}_\tau^4 &= &F_\tau^4\cdot(0,0,1).  \\
\end{alignedat}
\end{equation}

\begin{equation}
\label{eq:F_nu_i}
\begin{alignedat}{1}
\overline{F}_\nu^1 = F_\nu^1\cdot(1,0,0), \\
\overline{F}_\nu^2 = F_\nu^2\cdot(1,0,0), \\
\overline{F}_\nu^3 = F_\nu^3\cdot(1,0,0), \\
\overline{F}_\nu^4 = F_\nu^4\cdot(1,0,0). \\
\end{alignedat}
\end{equation}

Аналогично с пунктом \todo{вставить ссылку} выпишем уравнения статики и применим аналогичные допущения и симметриях сил трения и распределения нагрузок на ноги робота. В итоге получим следующую систему уравнений:

\begin{equation}
\label{eq:case2_final_equations}
\left\{
\begin{alignedat}{3}
  N_u - N_dk_\tau &= 0,\\
  2N_u + 2N_d k_\tau &= P,\\
  2 N_d h + P y_c &= 2 N_u (l + H k_\tau)\\
  \end{alignedat}
\right.
\end{equation}


Из системы \ref{eq:case2_final_equations} находим выражение для  $k$. Возможны два решения:

\begin{equation}
\label{eq:case2_k12}
    \begin{alignedat}{3}    
%      k = 
%      /                2            2        2             \
%      |  H - h + sqrt(H  - 2 H h + h  - 4 y_c  + 4 l y_c)  |
%      |  ------------------------------------------------  |
%      |                        2 y_c                       |
%      |                                                    |
%      |                 2            2        2            |
%      |   h - H + sqrt(H  - 2 H h + h  - 4 y_c  + 4 l y_c) |
%      | - ------------------------------------------------ |
%      \                         2 y_c                      /    
      k_1 &= \dfrac{(H-h) + \sqrt{(H-h)^2 - 4y^2_c + 4ly_c}}{2y_c}, \\
      k_2 &= \dfrac{(H-h) - \sqrt{(H-h)^2 - 4y^2_c + 4ly_c}}{2y_c} \\
    \end{alignedat}
\end{equation}

По условию задачи, нам подходит только $k_2$. Проверим решение $k_2$ на выполнение условия $0 < k_2 < 1$. Сначала рассмотрим условие что $k_2 > 0$:


\[
\begin{alignedat}{3}
\label{eq:case2_greater_than_zero}
0 &< k_2 = - \dfrac{(H - h) - \sqrt{(H-h)^2 - 4y_c^2+4ly_c}}{2y_c} \\
\dfrac{(H-h)}{2y_c} &< \dfrac{\sqrt{(H-h)^2 - 4y_c^2+4ly_c}}{2y_c}\\
\dfrac{(H-h)^2}{4y_c^2} &< \dfrac{(H-h)^2}{4y_c^2} - 1 + \dfrac{l}{y_c}\\
1 &< \dfrac{l}{y_c}\\
l &< y_c
\end{alignedat}
\]

По условию расположения опорных точек во второй конфигурации всегда верно $y_c > l$. Таким образом, $k_2$ из \ref{eq:phase_2_k12} всегда больше нуля. Найдем при каких условиях, $k_2$ меньше единицы, т.е. при каких условиях конфигурация робота будет статически устойчивой за счет кулоновского трения.

Решим следующее неравенство относительно переменных $y_c, l, (H-h)$:

\begin{equation}
\label{eq:case2_k_less_than_1}
\begin{alignedat}{3}
 - \dfrac{(H - h) - \sqrt{(H-h)^2 - 4y_c^2+4ly_c}}{2y_c}  & \leq 1 \\
\end{alignedat}
\end{equation}

Для этого введем безразмерные параметры:

\begin{equation}
\label{eq:case2_dimentionless}
p_1 := \dfrac{(H-h)}{y_c}, p_2 := \dfrac{l}{y_c}.
\end{equation}

После подстановки замены \ref{eq:case2_dimentionless} в неравенство \ref{eq:case2_k_less_than_1} получаем:

\begin{equation}
\label{eq:case2_k_less_one}
\begin{alignedat}{3}
  0 < \dfrac{1}{2}p_1 - \sqrt{\dfrac{1}{4}p_1^2 - 1 + p_2} < 1\\
  \dfrac{1}{4}p_1^2 - p_1 + 1 < \dfrac{1}{4}p_1^2 - 1 + p_2 \\
  p_2 < 2 - p_1\\
\end{alignedat}
\end{equation}


На координатной плоскости c осями $(p_1, p_2)$ решения неравенств \ref{eq:case2_k_less_than_1} и \ref{eq:case2_greater_than_zero} соответствует заштрихованной области:

\todo{вставить рисунок области}


Полученны условия на положение $y_c$ центра масс при заданных точках опоры используются для управления движением робота.


Для второй конфигурации осталось рассмотреть случай когда $y_c = 0$. Подставим $y_c = 0$ в уравнения \ref{eq:case2_final_equations}. Получим:

\begin{equation}
  \label{eq:case2_zero_yc}
    \left\{
    \begin{alignedat}{3}
      N_d - N_uk_\tau &= 0 \\
      2N_u+2N_dk_\tau &= P \\
      N_dh - N_ul &= HN_uk_\tau\\
    \end{alignedat}
    \right.
\end{equation}

Из системы \ref{eq:case2_zero_yc} однозначно получаем:

\begin{equation}
  \label{eq:case2_zero_yc_solution}
  \begin{alignedat}{3}
  k = -\dfrac{l}{H-h},\\
  N_u = \dfrac{P(H-h)^2}{2((H-h)^2+l^2)}\\
  N_d = \dfrac{-Pl(H - h)}{2((H-h)^2+l^2)}\\
  \end{alignedat}
\end{equation}


Таким образом, задача о проверке статической устойчивости шестиногого аппарата свелась к системе из трех уравнений с тремя неизвестными. Рассмотренные конфигурации покрывают все случа расстановки ног, за исключением конфигураций, когда выполняется смена опорных ног или когда аппарат опирается ногами только на горизонтальные площадки.