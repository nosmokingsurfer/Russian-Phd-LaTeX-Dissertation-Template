\chapter{Мозаичный корпус}

В данной главе рассматриваются аппараты с избыточным количеством степеней свободы и замкнутыми кинематическими цепями в корпусе. Представленный на рис. (  ) 
корпус аппарата назван мозаичным.

вставить рисунок: Многоногий шагающий аппарат с мозаичным корпусом

Изображенный на рисунке () робот состоит из девяти квадратных звеньев, составляющих корпус аппарата, и двенадцати инсектоморфных ног. Ноги крепятся при помощи одностепенных вращательных шарниров в вершинах звеньев корпуса. Квадратные звенья корпуса соединены между собой при помощи трехстепенных вращательных шарниров. К угловым сегментам корпуса крепится пара ног, к средним звеньям на каждой из четырех сторон корпуса крепится по одной ноге. К центральному звену корпуса ноги не присоединяются. Аналогом походки «трешками» шестиногого робота [1] для данного двенадцатиногого робота может быть походка «шестерками». Для аппарата на рис. 1 можно разделить ноги на две равные группы, так чтобы аппарат сохранял статическую устойчивость во время опоры на каждую группу ног.

\section{Кинематика корпуса}
Выделим структурные элементы корпуса. Рассмотрим часть корпуса, состоящую из четырех квадратных звеньев, соединенных в замкнутую цепь ( ). Квадратные звенья имею общие вершины $J_1, J_2, J_3$ и $J_4$, в которых расположены трехстепенные вращательные шарниры с углами поворота $(\psi_i,\theta_i,\phi_i)$ -самолетными углами Крылова.

вставить рисунок: Структурный элемент мозаичного корпуса

Для шарнира $J_i$ рис. () ось угла курса $\psi_i$ направлена вдоль оси $O_{i-1}z$, ось вращения угла тангажа $\theta_i$ направлена вдоль новой оси $O_{i-1}x'$, ось угла крена $\phi_i$ направлена вдоль новой оси $O_{i-1}y''$ --- ребра $J_iJ_{i+1}$ $i$-го сегмента.

Каждое звено ячейки $2x2$ поворачивается относительно других звеньев при помощи пары шарниров (рис.). При данном повороте в шарнире $J_i$ изменяется только угол $\phi_i$, а в шарнире $J_{i+1}$ происходит изменение всех шарнирных углов. Найдем как связан угол поворота $\phi_i$ для $i$-го звена и шарнирные углы $(\psi_{i+1},\theta_{i+1},\phi_{i+1})$.

вставить рисунок: Поворот звена, смежного с двумя фиксированными звеньями

Рассмотрим поворот $i$-го звена корпуса при фиксированных звеньях с номерами $i-1$ и $i+1$ рис(). При помощи матриц поворотов построим цепочку преобразований, связывающих системы координат $i-1$-го и $i+1$-го	звена. Пусть в начальный момент все шарнирные углы имели некоторые известные значения. Затем, $i$-ое звено совершило поворот на некоторый угол $\delta{\phi}_i:=\tilde{\phi}_i-\phi_i$ вокруг ребра $J_iJ_{i+1}$. Столбцы матрицы углов Крылова $R(\psi_i,\theta_i,\phi_i)$ записываются в виде:


\begin{array}{c}
	\left(
	\begin{array}{c}
	a_{11}\\
	a_{21}\\
	a_{31}\\
	end{array}
	\right) = 
	\left(
	\begin{array}{c}
	cos(\phi_i)\,cos(\psi_i)-sin(\phi_i),\sin(\psi_i)\,sin(\theta_i)\\
	cos(\phi_i)\,sin(\psi_i)+sin(\phi_i)\,cos(\psi_i)\,sin(\theta_i)\\
	-cos(\theta_i)\,sin(\phi_i)\\
	\end{array}
	\right)\\
	
\end{array}





\chapter{Вёрстка таблиц} \label{chapt3}

\section{Таблица обыкновенная} \label{sect3_1}

Так размещается таблица:

\begin{table} [htbp]
  \centering
  \parbox{15cm}{\caption{Название таблицы}\label{Ts0Sib}}
%  \begin{center}
  \begin{tabular}{| p{3cm} || p{3cm} | p{3cm} | p{4cm}l |}
  \hline
  \hline
  Месяц   & \centering $T_{min}$, К & \centering $T_{max}$, К &\centering  $(T_{max} - T_{min})$, К & \\
  \hline
  Декабрь &\centering  253.575   &\centering  257.778    &\centering      4.203  &   \\
  Январь  &\centering  262.431   &\centering  263.214    &\centering      0.783  &   \\
  Февраль &\centering  261.184   &\centering  260.381    &\centering     $-$0.803  &   \\
  \hline
  \hline
  \end{tabular}
%  \end{center}
\end{table}

%\newpage
%============================================================================================================================

\section{Параграф - два} \label{sect3_2}

Некоторый текст.

%\newpage
%============================================================================================================================

\section{Параграф с подпараграфами} \label{sect3_3}

\subsection{Подпараграф - один} \label{subsect3_3_1}

Некоторый текст.

\subsection{Подпараграф - два} \label{subsect3_3_2}

Некоторый текст.

\clearpage